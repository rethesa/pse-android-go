\section{Zielbestimmung}
Die Go-App soll es Nutzern erleichtern sich in Gruppen zu organisieren
und Treffpunkte in Form von Ort und Uhrzeit festzulegen. Zu gegebenem Zeitpunkt werden die Standorte der Gruppenmitglieder und des Zielortes angezeigt.


Auf einer Karte lässt sich dabei sehen, wo sich der Zielort und die anderen Gruppenmitglieder befinden.

\subsection{Musskriterien}
%\textbf{Clientseitig}
\begin{itemize}
	\item Benutzeraccount anlegen/ löschen
	\item Gruppen erstellen/ löschen
	\item Benutzer in Gruppen einladen/ aus Gruppen entfernen
	\item Gruppen beitreten/ verlassen
	\item Treffpunkte festlegen/ ändern
	\item Go-Button aktivieren/ deaktivieren
	\item Standorte von Gruppenmitgliedern anzeigen
	\item Standort des Zielortes anzeigen
\end{itemize}
%\textbf{Serverseitig}
%\begin{itemize}
	%\item Verschlüsselung des Datenaustausches zwischen Client/ Server
	%\item Synchronisation zwischen Gruppenmitgliedern
	%\item Gruppen erstellen/ löschen
%\end{itemize}

\subsection{Wunschkriterien}
\begin{itemize}
	\item Änderung von Benutzer- und Gruppennamen
	\item Administratorrechte an Gruppenmitglieder übertragen
	\item Inaktive Benutzer und Gruppen löschen
	\item Go-Button und GPS-Tracking getrennt verwalten
	\item Benachrichtigungen bei Gruppenaktivität
	\item Erweiterte Verwaltung von Treffpunkten
\end{itemize}

\subsection{Abgrenzungskriterien}
\begin{itemize}
	\item GPS-Daten werden keinen Benutzern zugeordnet
	\item Es gibt keine Suchfunktion für Benutzer und Gruppen
	\item Beitreten zu Gruppen ohne den dazugehörigen Link ist nicht möglich
	\item Gruppenlinks können nicht wiederverwertet werden
	\item Abstimmungen über Ort und Uhrzeit eines Treffpunktes werden nicht unterstützt
	\item Angaben über Zu- und Absagen oder Verspätungen zu einem Treffpunkt sind nicht möglich
	\item Es lassen sich keine Nachrichten, Fotos oder andere Dateien über die App versenden
\end{itemize}
