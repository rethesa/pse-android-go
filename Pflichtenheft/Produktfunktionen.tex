<<<<<<< Updated upstream
\section{Produktfunktionen/ Systemmodelle}
Funktionsübersicht: \\
/F10/ Erstmaliges Öffnen der App (Registrierung) Tarek \\
/F20/ Gruppe erstellen Dennis \\
/F30/ Mitglieder einladen Theresa \\
/F40/ Gruppe beitreten Dennis \\
/F50/ Treffpunkt (Ort/Zeit) festlegen (Admin) Vicky \\
/F60/ Gemeinsames Losgehen Matthias \\
/F70/ Gruppe verlassen Vicky \\
/F80/ Gruppe löschen Tarek \\
=======
\section{Funktionale Anforderungen}
\begin{description}
\item[$\bullet$] Registrierung von Benutzern durch einmalige Namensfestlegung
\item[$\hookrightarrow$] Name auf eindeutige ID abbilden und (Name,ID) an Server senden
\item[$\bullet$] Löschen des eigenen Nutzeraccounts
\item[$\hookrightarrow$] App auf anfangszustand zurücksetzen, Nutzer aus allen Gruppen entfernen und das Tupel (Name, ID) vom Server löschen
\item[$\bullet$] Erstellen von Gruppen mit eindeutigem Namen. Benutzer der die Gruppe erstellt wird zum Gruppenadministrator
\item[$\hookrightarrow$] Auf dem Server wird eine neue Gruppe angelegt mit Gruppenadministrator als einzigem Mitglied
\item[$\bullet$] Löschen von Gruppen (nur der Gruppenadministrator)
\item[$\hookrightarrow$] Entferne alle Mitglieder aus der Gruppe und Lösche Gruppe vom Server
\item[$\bullet$] Erstellen von Gruppenlinks. Diese werden als Einladung an zukünftige Mitglieder über externe Messenger versendet (nur Admin)
\item[$\hookrightarrow$] Gruppenlink wird zum Server gesendet und in Gruppe gespeichert. Nach der Nutzung wird er gelöscht
\item[$\bullet$] Beitreten einer Gruppe nur über Einladungslink (Voraussetzung registriert)
     Der Link wird über die App geöffnet (Falls installiert)
\item[$\hookrightarrow$] Link wird an Server gesendet und die zugehörige Gruppe identifiziert.\\ Die ID des Benutzers wird dann der Gruppenliste hinzugefügt
\item[$\bullet$] Verlassen einer Gruppe (als Gruppenmitglied)
\item[$\hookrightarrow$] Die ID des Benutzers wird aus der Gruppenliste auf dem Server gelöscht. Die aktuelle Liste wird mit den verbleibenden Benutzern synchronisiert
\item[$\bullet$] Entfernen von Gruppenmitgliedern (nur Gruppenadministrator): entspricht verlassen einer Gruppe, nur erzwungen durch Gruppenadministrator
\item[$\bullet$] Festlegen von Treffpunkten: einstellen von Zielort und Uhrzeit\\(nur Gruppenadministrator)
\item[$\hookrightarrow$] Parameter (Uhrzeit/Zielort) wird an Server gesendet und in der Gruppe gespeichert. Dann wird der Parameter mit den Mitgliedern synchronisiert
\item[$\bullet$] Go-Button: durch drücken des Go-Buttons eigene Position an die Gruppe übermitteln
\item[$\hookrightarrow$] Position als GPS Koordinaten an Server senden, der sie dann an die Mitglieder verteilt
\item[$\bullet$] Anzeigen der gemittelten GPS-Daten der Gruppe auf der Karte; sind mehrere Mitglieder innerhalbt eines bestimmten Radius,
     werden sie als Gruppe angezeigt
\item[$\hookrightarrow$] Daten der einzelnen Mitglieder werden über den Server an die Gruppe verteilt
\end{description}
\subsection{Nicht funktionale Anforderungen}
\begin{description}
\item[$\bullet$] Aktivieren und Deaktivieren des GPS-Tracking unabhängig vom Go-Button
\item[$\bullet$] Änderung des Benutzernamens nach erstmaliger Registrierung
\item[$\hookrightarrow$] Neuer Name muss mit dem Server synchronisiert werden bzw. mit allen Gruppen in denen der Benutzer ist
\item[$\bullet$] Gruppenmitglieder zu Administratoren machen
\item[$\hookrightarrow$] Dem Server mitteilen, dass bei gewünschtem Mitglied das Admin-Flag gesetzt werden soll.
\item[$\bullet$] Änderung des Gruppennamens nach erstellen der Gruppe
\item[$\hookrightarrow$] Neuer Name muss erst mit dem Server, dann mit allen Mitgliedern synchronisiert werden
\item[$\bullet$] Gruppen/Benutzer nach langer Inaktivität löschen
\item[$\hookrightarrow$] Der Server speichert den Zeitpunkt der letzten Aktivität von jedem Benutzer
\item[$\bullet$] Anzeigen von Gruppenmitgliedern, die den Go-Button schon gedrückt haben (Name Punkt auf Karte zuordnen)
\item[$\bullet$] Benachrichtigung über das 'Aufbrechen' anderer Gruppenmitglieder
\item[$\hookrightarrow$] Durch Drücken des Go-Buttons wird eine Benachrichtigung an den Server gesendet, der diese an die Mitglieder weiterleitet
\item[$\bullet$] Benachrichtigung bei ändern des Treffpunktes (Ort/Uhrzeit)
\item[$\hookrightarrow$] Gleiches Prinzip wie beim vorherigen Punkt
\item[$\bullet$] Treffpunkte für ein spezielles Datum festlegen
\end{description}
>>>>>>> Stashed changes


/F10/ \\
/F20/ \\
/F30/ \\ \\
Prozess: Mitglieder in eine Gruppe einladen\\
Ziel: Mitglieder erhalten einen Link zur Gruppe\\
Kategorie: primär\\
Vorbedingung: Gruppe wurde bereits erstellt\\
Nachbedingung (Erfolg): zukünftige Gruppenmitglieder erhalten einen Link\\
Nachbedingung (Fehlschlag): zukünftige Gruppenmitglieder erhalten keinen Link\\
Akteure: Gruppenadministrator\\
Auslösendes Ereignis: Ein oder mehrere Personen sind noch nicht Mitglieder der Gruppe\\
Beschreibung:\\
1.) Gruppenadministrator macht sich Gedanken welche Personen er in die Gruppe einladen möchte\\
2.) Er tippt auf den Namen der Gruppe und sieht die allgemeinen Informationen über die Gruppe (Namen, Mitglieder)\\
3.) Über "Share Link" kann er auswählen, über welchen externen Messenger er den Link weiterleiten möchte\\
4.) Er wählt den gewünschten Messenger und seine Kontakte und bestätigt\\
5.) Link wird an die ausgewählten Kontakte versendet\\ \\
/F40/ Prozess: Treffpunkt (Ort/Zeit) festlegen (Admin)\\
Ziel: Mitglieder erhalten definitive Angaben für den nächsten Termin\\
Kategorie: primär\\
Vorbedingung: feste Gruppe existiert, Mitglieder sind bestimmt, Administrator ist festgelegt\\
Nachbedingung (Erfolg): Alle Mitglieder können sehen, wann und wo (optional) das nächste Treffen stattfinden wird\\
Nachbedingung (Fehlschlag): Gruppenmitglied bekommt keine Einsicht zu den Daten des nächsten Treffens\\
Akteure: Gruppenadministrator\\
Auslösendes Ereignis: Ein neues Treffen ist in Planung\\
1.) Gruppenadministrator macht sich Gedanken, wann und wo das nächste Treffen der Gruppe stattfinden soll\\
2.) Er tippt auf den Namen der Gruppe und sieht die allgemeinen Informationen über die Gruppe (Namen, Mitglieder)\\
3.) Er tippt auf den Button "Ereignis erstellen"\\
4.) Er wird weitergeleitet zu einem neuen Fenster, in dem er aufgefordert wird Zeit und vorläufigen Treffpunkt des nächsten Treffens einzugeben\\
5.) nach Eingabe von mindestens einem der beiden Parameter, kann er das neue Treffen über einen Button "Treffen bestätigen" bestätigen\\
6.) nach Bestätigung des Treffens, wird ihm angezeigt\\
6.a) ob das Erstellen des neuen Treffens erfolgreich war
6.b) oder ob das Erstellen des neuen Treffens nicht erfolgreich war.
7.a) war das Erstellen des neuen Treffens erfolgreich, so wird er zurück geleitet auf die Gruppe und bekommt nun, wie alle Gruppenmitglieder die auf die Gruppe tippen, zusätzlich zu Namen und Mitglieder das neue Treffen angezeigt, das mit Zeit und Treffpunkt zu sehen ist\\
7.b) war das Erstellen des neuen Treffens nicht erfolgreich, so wird ihm in einem kleinen Fenster im Vordergrund angezeigt, welcher Fehler aufgetreten ist und er wird zur erneuten Eingabe aufgefordert. Durch drücken des Buttons "okay" wird das kleine Fenster geschlossen und er kommt zurück zu Punkt 4)\\ \\
/F50/ Prozess: Gruppe verlassen\\
Ziel: ein Mitglied verlässt eine aktive Gruppe\\
Kategorie: primär
Vorbedingung: User ist Mitglied in einer aktiven Gruppe
Nachbedingung (Erfolg): User ist kein Mitglied mehr der aktiven Gruppe\\
Nachbedingung (Fehlschlag): User ist weiterhin Mitglied der aktiven Gruppe\\
Akteure: Gruppenmitglied
Auslösendes Ereignis: User möchte aus einer aktiven Gruppe austreten\\
1.) User entscheidet sich, aus der Gruppe auszutreten\\
2.) Er tippt auf den Namen der Gruppe und sieht die allgemeinen Informationen über die Gruppe (Namen, Mitglieder)\\
3.) Er tippt auf den Button "Gruppe verlassen"\\
4.) Ihm wird ein kleines Fenster im Vordergrund angezeigt, das ihn erneut fragt, ob er diese Gruppe wirklich verlassen will\\
5.a) Er tippt auf den Button "Ja, Gruppe verlassen"\\
5.b) Er tippt auf den Button "Nein, in der Gruppe bleiben"\\
6.a) Er wird zurück geleitet auf die Startansicht mit der Übersicht über die Gruppen. Dabei ist die gelöschte Gruppe nicht mehr mit aufgelistet.\\
6.b) Das kleine Fenster wird geschlossen und er kommt zurück zu Punkt 2)\\
7.a) Bei allen weiteren Gruppenmitgliedern wird der Name dieses Users nicht mehr in der Liste der Gruppenmitglieder aufgeführt\\ 
/F60/ \\
/F70/ \\

/F80/ \\

\subsection{Funktionale Anforderungen}
aufgeteilt in Serverseitige und Nutzerseitige 
(und in Muss- und Wunschkriterien)
\subsection{Nichtfunktionale Anforderungen}