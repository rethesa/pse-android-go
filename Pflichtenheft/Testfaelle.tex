\section{Testfälle}
\subsection{Testfälle für Musskriterien}
%\begin{tabular}{ll}
\textbf{/T010/} Benutzer registrieren (FC010) \\
\begin{itemize}
Voraussetzung: der Benutzer ist mit dem Internet verbunden.\\
Der Benutzer gibt einen gültigen Benutzernamen ein.\\
Nach der Aktion liegt ein Tupel (Benutzername, ID) auf dem Server vor.\\
\end{itemize}

\textbf{/T020/} Benutzer registrieren Fehlschlag 1 (FC010) \\
\begin{itemize}
Voraussetzung: der Benutzer ist mit dem Internet verbunden.            \\
Der Benutzer gibt einen ungültigen Benutzernamen ein.                   \\
Nach der Aktion liegt kein Tupel (Benutzername, ID) auf dem Server vor.  \\
\end{itemize}

\textbf{/T030/} Benutzer registrieren Fehlschlag 2 (FC010) \\
\begin{itemize}
Voraussetzung: der Benutzer ist nicht mit dem Internet verbunden.         \\
Der Benutzer gibt einen Benutzernamen ein.                                 \\
Nach der Aktion liegt kein Tupel (Benutzername, ID) auf dem Server vor.     \\
\end{itemize}

\textbf{/T040/} Benutzeraccount löschen (FC020) \\
\begin{itemize}
Voraussetzung: der Benutzer ist mit dem Internet verbunden.               \\
Der Benutzer wählt aus, seinen Account zu löschen.                         \\
Nach der Ausführung ist das Tupel (Benutzername, ID) vom Server gelöscht.   \\
\end{itemize}

\textbf{/T050/} Benutzeraccount löschen Fehlschlag (FC020) \\
\begin{itemize}
Voraussetzung: der Benutzer ist nicht mit dem Internet verbunden.            \\
Der Benutzer wählt aus, seinen Account zu löschen.                            \\
Nach der Ausführung besteht das Tupel (Benutzername, ID) weiterhin auf dem Server.\\
\end{itemize}

\textbf{/T060/} Erstellen einer Gruppe (FC030)\\
\begin{itemize}
Voraussetzung: der Benutzer ist mit dem Internet verbunden.                        \\
Der Benutzer wählt Gruppe erstellen im Menü aus und gibt einen Namen ein.           \\
Als Folge liegt auf dem Server eine Gruppe vor, mit dem Ersteller als einzigem Mitglied.\\
\end{itemize}

\textbf{/T070/} Erstellen einer Gruppe Fehlschlag (FC030)\\
\begin{itemize}
Voraussetzung: der Benutzer ist nicht mit dem Internet verbunden.              \\
Der Benutzer wählt Gruppe erstellen im Menü aus und gibt einen Namen ein.       \\
Es wird keine neue Gruppe erzeugt, die auf dem Server auch nicht vorliegt.       \\
\end{itemize}

\textbf{/T080/} Löschen einer Gruppe (FC040)\\
Voraussetzung: der Benutzer ist mit dem Internet verbunden.                     \\
Der Benutzer wählt aus, eine seiner selbst erstellten Gruppen zu löschen.        \\
Danach ist die Gruppe sowohl vom Server, als auch bei den Mitgliedern gelöscht.   \\
\begin{itemize}

\textbf{/T090/} Löschen einer Gruppe Fehlschlag (FC040)\\
Voraussetzung: der Benutzer ist nicht mit dem Internet verbunden.                 \\
Der Benutzer wählt aus, eine seiner selbst erstellten Gruppen zu löschen.          \\
Die Gruppe bleibt sowohl beim Benutzer, als auch auf dem Server bestehen.           \\
\begin{itemize}

\textbf{/T100/} Erstellen eines Gruppenlinks (FC050)\\
Der Benutzer hat eine Gruppe erstellt und will Mitglieder einladen.      \\
Er ist mit dem Internet verbunden.                                        \\
Er wählt im Gruppenmenü der entsprechenden Gruppe "Mitglieder einladen" aus. \\
Der Benutzer erhält einen Gruppenlink, der auch in der Gruppe gespeichert ist.\\
\begin{itemize}

\textbf{/T110/} Erstellen eines Gruppenlinks Fehlschlag (FC050)\\
Der Benutzer hat eine Gruppe erstellt und will Mitglieder einladen.            \\
Er ist nicht mit dem Internet verbunden.                                        \\
Er wählt im Gruppenmenü der entsprechenden Gruppe "Mitglieder einladen" aus.     \\
Der Benutzer erhält keinen Gruppenlink und in der Gruppe auf dem Server ist auch kein neuer Link.\\
\begin{itemize}

\textbf{/T120/} Beitreten einer Gruppe (FC060)\\
\begin{itemize}
Voraussetzung: Gruppe besteht, Benutzer ist registriert und mit dem Internet verbunden. \\
Der Benutzer bekommt einen Link zugesendet und öffnet ihn mit der App.                   \\
Nach öffnen des Links, ist der Benutzer in der Liste der entsprechenden Gruppe.           \\
Außerdem ist der genutzte Link aus der Gruppe gelöscht.                                    \\
\end{itemize}

\textbf{/T130/} Beitreten einer Gruppe Fehlschlag (FC060)\\
\begin{itemize}
Voraussetzung: Gruppe besteht, Benutzer ist registriert aber nicht mit dem Internet verbunden.\\
Der Benutzer hat einen Link erhalten und öffnet ihn mit der App.                               \\
Nach öffnen des Links, ist der Benutzer nicht in der Liste der entsprechenden Gruppe.           \\
Der Link bleibt in der Gruppe auf dem Server gespeichert.                                        \\
\end{itemize}

\textbf{/T140/} Verlassen einer Gruppe (FC070)\\
\begin{itemize}
Voraussetzung: Benutzer ist mitglied in der Gruppe und mit dem Internet verbunden.\\
Benutzer wählt "Gruppe verlassen" aus.\\
Nach der Aktion ist der Benutzer nicht mehr in der Gruppenliste auf dem Server.\\
Zudem ist die Gruppe nicht mehr in der Liste des Benutzers.\\
\end{itemize}

\textbf{/T150/} Verlassen einer Gruppe Fehlschlag (FC070)\\
\begin{itemize}
Voraussetzung: Benutzer ist mitglied in der Gruppe aber nicht mit dem Internet verbunden.\\
Benutzer wählt "Gruppe verlassen" aus.\\
Nach der Aktion verbleibt der Benutzer in der Gruppenliste auf dem Server.\\
Zudem ist die Gruppe immernoch in der Liste des Benutzers.\\
\end{itemize}

\textbf{/T160/} Entfernung von Gruppenmitgliedern (FC080)\\
\begin{itemize}
Voraussetzung: Benutzer ist Admin in der Gruppe und mit dem Internet verbunden.\\
In der Gruppe befindet sich ein Mitglied das entfernt werden soll.              \\
Der Benutzer wählt bei einem Mitglied "Mitglied entfernen" aus.                  \\
Nach der Ausführung ist das Mitglied nicht mehr in seiner Liste.                 \\
Auf dem Server ist das Mitglied aus der Gruppenliste gelöscht.                    \\
\end{itemize}

\textbf{/T170/} Entfernung von Gruppenmitgliedern Fehlschlag (FC080)\\
\begin{itemize}
Voraussetzung: Benutzer ist Admin in der Gruppe aber nicht mit dem Internet verbunden.\\
In der Gruppe befindet sich ein Mitglied das entfernt werden soll.\\
Der Benutzer wählt bei einem Mitglied "Mitglied entfernen" aus.    \\
Nach der Ausführung bleibt das gewählte Mitglied in seiner Liste.   \\
Auf dem Server ist das Mitglied immernoch in der Gruppenliste.       \\
\end{itemize}

\textbf{/T180/} Festlegen des Zielorts (FC090)\\
\begin{itemize}
Voraussetzung: Benutzer ist Admin in der Gruppe und mit dem Internet verbunden.\\
Der Benutzer editiert den Zielort der Gruppe.                                   \\
Nach der Ausführung ist der neue Zielort in der Gruppe sowohl beim Benutzer,     \\
als auch auf dem Server gespeichert.                                              \\
\end{itemize}

\textbf{/T190/} Festlegen der Uhrzeit (FC100)\\
\begin{itemize}
Voraussetzung: Benutzer ist Admin in der Gruppe und mit dem Internet verbunden.    \\
Der Benutzer editiert die Uhrzeit der Gruppe.                                       \\
Nach der Aktion ist die neue Uhrzeit beim Benutzer und auf dem Server gespeichert.   \\
\end{itemize}

\textbf{/T200/} Festlegen eines Gruppenparameters Fehlschlag (FC100 / FC090)\\
\begin{itemize}
Voraussetzung: Benutzer ist Admin in der Gruppe aber nicht mit dem Internet verbunden.\\
Der Benutzer editiert Uhrzeit/Zielort der Gruppe.                                      \\
Nach der Aktion ist die alte Uhrzeit/ der alte Zielort beim Benutzer und auf dem Server unverändert.\\
\end{itemize}

\textbf{/T210/} Drücken des Go-Buttons (FC110)\\
\begin{itemize}
Voraussetzung: Benutzer ist in der Gruppe und mit dem Internet verbunden, GPS aktiviert.\\
Der Benutzer drückt den Go-Button und seine GPS-Daten werden vom Server empfangen.       \\
\end{itemize}

\textbf{/T220/} Drücken des Go-Buttons ohne Internetverbindung (FC110)\\
\begin{itemize}
Voraussetzung: Benutzer ist in der Gruppe und nicht mit dem Internet verbunden.           \\
Der Benutzer drückt den Go-Button aber seine GPS-Daten werden nicht vom Server empfangen.  \\
\end{itemize}

\textbf{/T230/} Erneutes Drücken des Go-Buttons (FC120)\\
\begin{itemize}
Voraussetzung: Benutzer ist in der Gruppe und mit dem Internet verbunden, GPS aktiviert.    \\
Der Benutzer drückt den Go-Button ein zweites Mal und seine GPS-Daten werden nicht mehr vom Server empfangen.\\
\end{itemize}

\textbf{/T240/} Anzeigen der GPS-Daten auf der Karte (FC130)\\
\begin{itemize}
Voraussetzung: Benutzer ist in der Gruppe und mit dem Internet verbunden.                                     \\
Der Benutzer empfängt GPS-Daten vom Server. Diese werden bei ausreichend geringem Abstand gemittelt            \\
und auf der Karte als einzelner Punkt angezeigt.                                                                \\
\end{itemize}

\textbf{/T250/} Verschlüsselung zwischen Client und Server (FC140 / FS130)\\
\begin{itemize}
Voraussetzung: Benutzer ist in der Gruppe und mit dem Internet verbunden.            \\
Von der App an das Internet gehende Daten sind verschlüsselt.                         \\
Vom Server ausgehende Daten sind ebenfalls verschlüsselt.                              \\
\end{itemize}

\textbf{/T260/} Abrufen aktueller Gruppenparameter (FC150)\\
\begin{itemize}
Voraussetzung: Benutzer ist in der Gruppe und mit dem Internet verbunden.     \\
Der Benutzer hat die App geöffnet.                                             \\
Beim Server geht eine Anfrage ein über aktuelle Gruppenparameter des Benutzers. \\
\end{itemize}

\textbf{/T270/} Abrufen aktueller Gruppenparameter ohne Internetverbindung (FC150)\\
\begin{itemize}
Voraussetzung: Benutzer ist in der Gruppe aber nicht mit dem Internet verbunden. \\
Der Benutzer hat die App geöffnet.                                                \\
Beim Server gehen keine Anfragem ein über aktuelle Gruppenparameter des Benutzers. \\
\end{itemize}

\subsection{Zu den Wunschkriterien:}\\

\textbf{/T280/} Drücken des Go-Buttons ohne aktiviertem GPS (FC160)\\
\begin{itemize}
Voraussetzung: Benutzer ist in Gruppe und ist mit dem Internet verbunden.           \\
Benutzer drückt Go-Button, beim Server wird empfangen, dass der Benutzer den Go-Button\\
gedrückt hat, aber keine GPS-Daten sendet.                                             \\
\end{itemize}

\textbf{/T290/} Ändern des Benutzernamens (FC170)\\
\begin{itemize}
Voraussetzung: der Benutzer hat bereits einen Benutzernamen festgelegt und ist mit dem Internet verbunden.\\
Benutzer verändert seinen Benutzernamen.                                                                   \\
Auf dem Server ist auf die alte ID des Benutzers der neue Name abgebildet.                                  \\
\end{itemize}

\textbf{/T300/} Ernennen weiterer Gruppenadministratoren (FC180)\\
\begin{itemize}
Voraussetzung: Benutzer ist Admin in Gruppe und ist mit dem Internet verbunden.\\
Benutzer wählt ein Gruppenmitglied aus, das zu einem weiteren Gruppenadministrator werden soll.\\
Auf dem Server ist in der Gruppenliste vermerkt, dass das Mitglied ebenfalls Administrator ist.\\
\end{itemize}

\textbf{/T310/} Umbenennen der einer Gruppe (FC190)\\
\begin{itemize}
Voraussetzung: Benutzer ist Admin in Gruppe und ist mit dem Internet verbunden.\\
Der Benutzer ändert den Gruppennamen.\\
Auf dem Server liegt die alte Gruppe mit neuem Namen vor.\\
\end{itemize}

\textbf{/T320/} Benutzer nach langer Inaktivität löschen (FC200)\\
\begin{itemize}
Voraussetzung: Benutzer ist registriert.\\
Der Benutzer war länger als eine bestimmte Zeit nicht mehr aktiv.\\
Er ist nach der Zeit nicht mehr auf dem Server vorhanden.\\
\end{itemize}

\textbf{/T330/} Letztes Mitglied einer Gruppe nach langer Inaktivität löschen (FC200)\\
\begin{itemize}
Voraussetzung: Benutzer ist registriert und einziges Mitglied einer Gruppe.\\
Der Benutzer war länger als eine bestimmte Zeit nicht mehr aktiv.\\
Er ist nach dieser Zeit nicht mehr auf dem Server gespeichert. Die leer gewordene Gruppe\\
ist ebenfalls gelöscht.\\
\end{itemize}

\textbf{/T340/} Anzeigen der Namen von Gruppenmitgliedern die den Go-Button gedrückt haben (FC210)\\
\begin{itemize}
Voraussetzung: Benutzer ist in Gruppe, mind. ein anderes Gruppenmitglied drückt Go-Button.\\
Der Benutzer empfängt neben den GPS-Daten auch den Benutzernamen des Mitglieds, das "Go" gedrückt hat.\\
\end{itemize}

\textbf{/T350/} Benachrichtigung bei Aufbruch anderer Gruppenmitglieder (FC220)\\
\begin{itemize}
Voraussetzung: Benutzer ist in Gruppe, mind. ein anderes Gruppenmitglied drückt Go-Button.\\
Der Benutzer erhält eine Nachricht auf seinem Gerät, dass das andere Gruppenmitglied "Go" gedrückt hat.\\
\end{itemize}

\textbf{/T360/} Benachrichtigung bei neuem Treffpunkt Parameter (FC220)\\
\begin{itemize}
Voraussetzung: Benutzer ist in Gruppe, der Gruppenadmin ändert einen Gruppenparameter.\\
Der Benutzer erhält eine Nachricht auf seinem Gerät, dass sich ein Gruppenparameter geändert hat.\\
\end{itemize}

%\end{tabular}
