\section{Qualitätsbestimmung}
\textbf{Produktqualität:}\\ %-- 2. wichtigste
% Funktionalität - Angemessenheit - Richtigkeit - Interoperabilität - Ordnungmäßigkeit - Sicherheit\\
Die Qualität der App ist auf einem hohen Niveau, sei es bei der Funktionalität oder der Richtigkeit. Die App soll solide Benutzer registrieren, Gruppen erstellen und das erstellen von Treffpunkten umsetzen. Ebenso ist uns die Datensicherheit in der App wichtig.\\
\textbf{Zuverlässigkeit:}\\% -- sehr wichtig
% Fehlertoleranz - Wiederherstellbarkeit\\
Unser wichtigstes Qualitätsziel ist die Zuverlässigkeit der Hauptfunktionen der App, wie oben aufgelistet. Unsere App soll unter Durchschnittsanfragen stabil laufen und eine große Fehlertoleranz haben. \\
\textbf{Benutzbarkeit:}\\ %-- 3. wichtigste
%Verständlichkeit - Erlernbarkeit - Bedienbarkeit\\
Die App ist übersichtlich und ähnlich wie bereits bekanntere Apps zu bedienen, wie zum Beispiel "Google Maps". Trotzdem ist die Bedienung auch für Kunden, die bisher kaum mit Android-Geräten gearbeitet haben, leicht ersichtlich.\\
\textbf{Effizienz:}\\ %-- nicht sooo wichtig
%Zeitverhalten - Verbrauchsverhalten\\
\\
\textbf{Änderbarkeit:}\\ %-- weniger wichtig
%Analysierbarkeit - Modifizierbarkeit - Stabilität - Prüfbarkeit\\
Die App ist sowohl prüfbar als auch analysierbar sein. \\
\textbf{Übertragbarkeit:}\\% -- weniger wichtig
%Anpassbarkeit - Installierbarkeit - Austauschbarkeit\\
Der Programm Code ist Erweiterbar.\\