\documentclass[parskip=full]{scrartcl}
\usepackage[utf8]{inputenc}
\usepackage[T1]{fontenc}
\usepackage[german]{babel}
\usepackage{hyperref}
\hypersetup{ 
pdftitle={
PSE: Pflichtenheft},
bookmarks=true,
}
\usepackage{csquotes}



\begin{document}
\title{Android Go-App}
\author{Tarek, \\Dennis, \\Matthias, \\Victoria Karl, \\Theresa Heine\\
	\\Betreuer: \\Heiko Klare, \\ Erik ...\\}	
\date{\today}
\maketitle
\newpage
\tableofcontents
\newpage


\section{Zielbestimmung}
\subsection{Musskriterien}

\subsection{Wunschkriterien}
\subsection{Abgrenzungskriterien}

\section{Produkteinsatz}
\subsection{Anwendungsbereiche}
\subsection{Zielgruppen}
\subsection{Betriebsbedingungen}

\section{Produktumgebung}
\subsection{Software}
\subsection{Hardware}

\section{Produktübersicht}

\section{Produktfunktionen}
Funktionsübersicht:
F10 Erstmaliges Öffnen der App (Registrierung) Tarek
F20 Gruppe erstellen Dennis
F30 Mitglieder einladen Theresa
F40 Gruppe beitreten Dennis
F50 Treffpunkt (Ort/Zeit) festlegen (Admin) Vicky
F60 Gemeinsames Losgehen Matthias
F70 Gruppe verlassen Vicky
F80 Gruppe löschen Tarek

\section{Produktdaten}
\subsection{Personendaten}
\subsection{title}

\section{Produktleistungen}

\section{Benutzerschnittstelle}

\section{Globale Testfälle}

\section{Qualitätsbestimmung}
Funktionalität 
Zuverlässigkeit
Benutzbarkeit 
Effizienz 
Änderbarkeit 
Übertragbarkeit

\section{A.Anhang}
Bilder und Flipcharts

\section{B.Anhang}
Glossar


\end{document}