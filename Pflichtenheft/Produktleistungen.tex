\section{Nicht funktionale Anforderungen}
%kleine einleitung zu kapitel?
%Die Nichtfunktionale Anforderungen beschreiben neben den Funktionalen Anforderungen, was das System zu leisten hat. Hier liegt der Schwerpunkt auf wie das System die Funktionen erfüllt. Beispielsweise wie groß ein System ist, den Durchsatz der Datenübertragung zum Server sowie die maximale Antwortzeiten. 
  
\textbf{Stabilität:}\\
\textbf{/NF010/}Es müssen insgesamt maximal 1000 Benutzer und 500 Gruppen verwaltet werden können. \\
\textbf{/NF020/}Zu einem Zeitpunkt können bis zu 100 Benutzer gleichzeitig die App ohne Problem benutzen \\
%/NF030/	Wenn kein Datenabgleich mit dem Server möglich ist, so werden die Daten dennoch auf den jeweiligen Geräten abgespeichert.	  \\

\textbf{Datensicherheit:}\\
\textbf{/NF030/}Personenbezogene Daten (Name und Standort) werden während der Datenübertragung verschlüsselt.\\
\textbf{/NF040/}Verschlüsselungsdaten müssen klein gehalten werden, um System Skalierbarkeit zu ermöglichen.\\

\textbf{Benutzbarkeit:}\\
\textbf{/NF050/}Die "Go-App" hat eine intuitive Benutzeroberfläche.\\
%/NF0X0/	Die "Go-App" sollte flüssig laufen.\\
\textbf{/NF060/}Die Antwortzeit der "Go-App" beträgt im durchschnitt ungefähr 2 Sekunden haben.\\

\textbf{Effizienz:}\\
\textbf{/NF070/}Bei Eingang der Daten auf dem Server sollte die Gruppenerstellung im durchschnitt nicht länger als 5 Sekunden dauern\\
\textbf{/NF080/}Bei Eingang der Daten auf dem Server sollte die Registrierung im durchschnitt nicht länger als 5 Sekunden dauern.\\
\textbf{/NF090/}Die GPS Standort Übermittlung hat im schnitt eine Genauigkeit von ungefähr 15 Metern \\
\textbf{/NF100/}Die GPS Standort Verwaltung auf dem Server darf im schnitt nicht länger als 5 Sekunden dauern  \\
