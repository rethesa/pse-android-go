\section{Einleitung}
In unserer heutigen modernen Welt ist Planung das A und O. Sei es eine Wanderung im Wald, eine Kneipentour oder ein einfaches Mittagessen mit den Arbeitskollegen, um die Planung muss man sich kümmern! 
Nicht nur die Planung ist so wichtig, sondern auch die Zeit. Man möchte schnell und effektiv seine Vorhaben mit der Gruppe planen und verwalten.\\
Zum Beispiel einige Kollegen möchten ihr Mittagessen zusammen genießen, jedoch haben diese während der Arbeit kaum Zeit sich zu einigen. Deshalb ist es zu unserer Aufgabe geworden, diese kleine Treffen zu modernisieren und zu vereinfachen. Mit unserer Go-App ist es möglich eine Gruppe zu erstellen und seine Kollegen oder Freunde in diese einzuladen. Mit einem Klick werden alle über die erstellten Termine informiert . Ebenfalls soll die Wiederfindung vereinfacht werden um sich schnell und komfortabel an dem Treffpunkt zu finden.\\
Wir, das Team „Go-App“, haben es uns dabei zur Aufgabe gemacht, das Treffen einer Gemeinschaft zeitgemäß zu koordinieren. Dabei ist unser Projekt im Rahmen eines Software-Projektes am KIT entstanden. \\

%KOMMENTAR:
%- [DONE] - zu Nebensatz geändert   -"Sei es eine Wanderung im Wald, eine Kneipentour oder ein einfaches Mittagessen mit den Arbeitskollegen." ist kein echter Satz. da fehlt das Verb!!
%- du brauchst nicht jeden Satz zu betonen, dass es um die Planung geht.
%- Vermeide ebenfalls dopplungen wie "wo und wann"
%- auch nicht "so xxx wie möglich"  2x hintereinander bringen 
%- "Man möchte so schnell wie möglich und so effektiv wie möglich PLANUNG erstellen und sich daran halten." das ist doch gar nicht der Sinn der App. Vielleicht eher: Man möchte so schnell und komfortabel 	wie möglich planen können ohne sich zu sehr festlegen zu müssen.
%- Angenommen    -----> Wenn zum Beispiel
%- Arbeit        -----> Arbeitszeit
%- versuch das "man" zu vermeiden. Das kannst du ein mal bringen, aber nicht ständig... z.B. "wo und wann sie essen wollen"     .... "wo und wann die Gruppe essen wird"
%- Gruppe ist auch zu häufig hintereinander
%- trenne das informiert werden und Standort anzeigen gedanklich in 2 Sätze auf... "Mit einem Klick (ein Klick ist immer einfach) werden alle Gruppenmitglieder darüber informiert, wo und wann die Gruppe essen wird. Mit einem weiteren Klick zeigen machen alle Gruppenmitglieder ihren aktuellen Standort sichtbar, womit weder treffe noch verspäten mehr ein Problem darstellt."
%- der letzte Satz fällt vom Himmel. Entweder du machst da noch einen schönen Übergang, wurschtelst den oben noch mit rein, oder löscht ihn (was schade wäre, denn prinzipiell ist er nett)
%- außerdem haben wir es uns wirklich zur Aufgabe gemacht "die Planung zu vereinfachen"? oder geht es nicht viel mehr darum sich leichter zu finden?