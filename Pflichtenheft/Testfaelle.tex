\section{Globale Testfälle}
\subsection{Testfälle für Musskriterien}
%\begin{tabular}{ll}
\textbf{/T010/} Benutzer registrieren \\
\begin{itemize}
\setlength{\itemsep}{0pt}
\item Der Benutzer hat die App installiert. Dies wird über den Android Package Manager geprüft.
\item Der Benutzer registriert einen Account. Dies wird in der Datenbank des Servers geprüft.
\item Im Fehlerfall wird dem Benutzer eine Fehlermeldung angezeigt.
\end{itemize}


\textbf{/T020/} Benutzer löschen \\
\begin{itemize}
\setlength{\itemsep}{0pt}
\item Der Benutzer ist registriert. Dies wird über die auf dem Gerät
und Server gespeicherten Daten geprüft.
\item Der Benutzer löscht seinen Account. Dies wird in der Datenbank des Servers geprüft.
\item Im Fehlerfall wird dem Benutzer eine Fehlermeldung angezeigt.
\end{itemize}


\textbf{/T030/} Gruppe erstellen \\
\begin{itemize}
\setlength{\itemsep}{0pt}
\item Der Benutzer ist registriert. Dies wird über die auf dem Gerät
und Server gespeicherten Daten geprüft.
\item Der Benutzer erstellt eine Gruppe. () Dies wird in der Datenbank des Server geprüft.
\item Im Fehlerfall wird dem Benutzer eine Fehlermeldung angezeigt.
\end{itemize}

\textbf{/T040/} Gruppe löschen \\
\begin{itemize}
\setlength{\itemsep}{0pt}
\item Der Benutzer ist Administrator der Gruppe. Dies wird in der Datenbank des Servers geprüft.
\item Der Benutzer löscht die Gruppe. () Dies wird in der Datenbank des Servers geprüft.
\item Im Fehlerfall wird dem Benutzer eine Fehlermeldung angezeigt.
\end{itemize}

\textbf{/T050/} Einladung erstellen \\
\begin{itemize}
\setlength{\itemsep}{0pt}
\item Der Benutzer ist Administrator der Gruppe. Dies wird in der Datenbank des Servers geprüft.
\item Der Benutzer erstellt () und verschickt Einladungen. Dies wird in der Datenbank des Servers geprüft.
\item Im Fehlerfall wird dem Benutzer eine Fehlermeldung angezeigt.
\end{itemize}


\textbf{/T060/} Gruppe beitreten \\
\begin{itemize}
\setlength{\itemsep}{0pt}
\item Der Benutzer wurde von einem Administrator zu einer Gruppe eingeladen. ()
\item Der Benutzer tritt der Gruppe () bei. Dies wird in der Datenbank des Servers geprüft.
\item Im Fehlerfall wird dem Benutzer eine Fehlermeldung angezeigt.
\end{itemize}


\textbf{/T070/} Gruppe verlassen \\
\begin{itemize}
\setlength{\itemsep}{0pt}
\item Der Benutzer ist Mitglied der Gruppe. Dies wird in der Datenbank des Servers geprüft.
\item Der Benutzer verlässt die Gruppe. () Dies wird in der Datenbank des Servers geprüft.
\item Im Fehlerfall wird dem Benutzer eine Fehlermeldung angezeigt.
\end{itemize}


\textbf{/T080/} Aus Gruppe entfernen \\
\begin{itemize}
\setlength{\itemsep}{0pt}
\item Der Benutzer ist Administrator der Gruppe. Dies wird in der Datenbank des Servers geprüft.
\item Der Benutzer wählt eine Mitglied zum Entfernen aus. Dies wird in der Datenbank des Servers geprüft.
\item Im Fehlerfall wird dem Benutzer eine Fehlermeldung angezeigt.
\end{itemize}


\textbf{/T090/} Treffpunkt festlegen \\
\begin{itemize}
\setlength{\itemsep}{0pt}
\item Der Benutzer ist Administrator der Gruppe. Dies wird in der Datenbank des Servers geprüft.
\item Der Benutzer legt einen Treffpunkt fest. () Dies wird in der Datenbank des Servers geprüft.
\item Im Fehlerfall wird dem Benutzer eine Fehlermeldung angezeigt.
\end{itemize}


\textbf{/T100/} Go-Button drücken / Aufbrechen \\ % TODO Formulierung?
\begin{itemize}
\setlength{\itemsep}{0pt}
\item Der Benutzer ist Mitglied einer Gruppe. Dies wird in der Datenbank des Servers geprüft.
\item Ein Administrator hat einen Treffpunkt festgelegt. () Dies wird in der Datenbank des Servers geprüft.
\item Der Benutzer bricht auf / drückt den Go-Button. ()
Seine Position wird übermittelt.
Dies wird beim Server geprüft.
\item Im Fehlerfall wird dem Benutzer eine Fehlermeldung angezeigt.
\end{itemize}


\textbf{/T110/} Positionen von Gruppenmitgliedern anzeigen \\
\begin{itemize}
\setlength{\itemsep}{0pt}
\item Der Benutzer ist Mitglied einer Gruppe. Dies wird in der Datenbank des Servers geprüft.
\item Mindestens ein Mitglied ist aufgebrochen. () Dies wird in der Datenbank des Servers geprüft.
\item Der Benutzer ruft die Karte auf und sieht die Positionen der
aufgebrochenen Mitglieder. () Dies wird auf dem Gerät überprüft.
\item Im Fehlerfall wird dem Benutzer eine Fehlermeldung angezeigt.
\end{itemize}


%\end{tabular}

\subsection{Testfälle für Wunschkriterien}
\begin{tabular}{ll}
/T200/ & Losgehen mit aktivierten/deaktiviertem GPS-Tracking \\ % TODO Formulierung Losgehen?
/T210/ & Änderung des Benutzernamens \\
/T220/ & Gruppenmitglied als zusätzlichen Administrator setzen \\
/T230/ & Änderung des Gruppennamens \\
/T240/ & Inaktive Gruppen löschen \\
/T250/ & Inaktive Benutzer löschen \\
/T260/ & Anzeigen des Aufbruchs-Status von Gruppenmitgliedern \\
/T270/ & Benachrichtigung über Aufbrechen von Gruppenmitgliedern \\
/T280/ & Benachrichtigung über Änderung des Treffpunktes \\
/T290/ & Festlegen von Treffpunkten für spezifische Zeitpunkte \\
\end{tabular}
