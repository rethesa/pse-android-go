
\section{Zielbestimmung}
Die App soll es Nutzern erleichtern sich in Gruppen zu organisieren
und Treffpunkte in Form von Zeit und Ort festzulegen.
\begin{description}
\item[$\bullet$]: Clientseitige Aktionen
\item[$\hookrightarrow$]: Serverseitige Aktionen
\end{description}
\subsection{Musskriterien}
\begin{description}
\item[$\bullet$] Registrierung von Benutzern durch einmalige Namensfestlegung
\item[$\hookrightarrow$] Name auf eindeutige ID abbilden und (Name,ID) an Server senden
\item[$\bullet$] Löschen des eigenen Nutzeraccounts
\item[$\hookrightarrow$] App auf anfangszustand zurücksetzen, Nutzer aus allen Gruppen entfernen und das Tupel (Name, ID) vom Server löschen
\item[$\bullet$] Erstellen von Gruppen mit eindeutigem Namen. Benutzer der die Gruppe erstellt wird zum Gruppenadministrator
\item[$\hookrightarrow$] Auf dem Server wird eine neue Gruppe angelegt mit Gruppenadministrator als einzigem Mitglied
\item[$\bullet$] Löschen von Gruppen (nur der Gruppenadministrator)
\item[$\hookrightarrow$] Entferne alle Mitglieder aus der Gruppe und Lösche Gruppe vom Server
\item[$\bullet$] Erstellen von Gruppenlinks. Diese werden als Einladung an zukünftige Mitglieder über externe Messenger versendet (nur Admin)
\item[$\hookrightarrow$] Gruppenlink wird zum Server gesendet und in Gruppe gespeichert. Nach der Nutzung wird er gelöscht
\item[$\bullet$] Beitreten einer Gruppe nur über Einladungslink (Voraussetzung registriert)
     Der Link wird über die App geöffnet (Falls installiert)
\item[$\hookrightarrow$] Link wird an Server gesendet und die zugehörige Gruppe identifiziert.\\ Die ID des Benutzers wird dann der Gruppenliste hinzugefügt
\item[$\bullet$] Verlassen einer Gruppe (als Gruppenmitglied)
\item[$\hookrightarrow$] Die ID des Benutzers wird aus der Gruppenliste auf dem Server gelöscht. Die aktuelle Liste wird mit den verbleibenden Benutzern synchronisiert
\item[$\bullet$] Entfernen von Gruppenmitgliedern (nur Gruppenadministrator): entspricht verlassen einer Gruppe, nur erzwungen durch Gruppenadministrator
\item[$\bullet$] Festlegen von Treffpunkten: einstellen von Zielort und Uhrzeit\\(nur Gruppenadministrator)
\item[$\hookrightarrow$] Parameter (Uhrzeit/Zielort) wird an Server gesendet und in der Gruppe gespeichert. Dann wird der Parameter mit den Mitgliedern synchronisiert
\item[$\bullet$] Go-Button: durch drücken des Go-Buttons eigene Position an die Gruppe übermitteln
\item[$\hookrightarrow$] Position als GPS Koordinaten an Server senden, der sie dann an die Mitglieder verteilt
\item[$\bullet$] Anzeigen der gemittelten GPS-Daten der Gruppe auf der Karte; sind mehrere Mitglieder innerhalbt eines bestimmten Radius,
     werden sie als Gruppe angezeigt
\item[$\hookrightarrow$] Daten der einzelnen Mitglieder werden über den Server an die Gruppe verteilt
\end{description}
\subsection{Wunschkriterien}
\begin{description}
\item[$\bullet$] Aktivieren und Deaktivieren des GPS-Tracking unabhängig vom Go-Button
\item[$\bullet$] Änderung des Benutzernamens nach erstmaliger Registrierung
\item[$\hookrightarrow$] Neuer Name muss mit dem Server synchronisiert werden bzw. mit allen Gruppen in denen der Benutzer ist
\item[$\bullet$] Gruppenmitglieder zu Administratoren machen
\item[$\hookrightarrow$] Dem Server mitteilen, dass bei gewünschtem Mitglied das Admin-Flag gesetzt werden soll.
\item[$\bullet$] Änderung des Gruppennamens nach erstellen der Gruppe
\item[$\hookrightarrow$] Neuer Name muss erst mit dem Server, dann mit allen Mitgliedern synchronisiert werden
\item[$\bullet$] Gruppen/Benutzer nach langer Inaktivität löschen
\item[$\hookrightarrow$] Der Server speichert den Zeitpunkt der letzten Aktivität von jedem Benutzer
\item[$\bullet$] Anzeigen von Gruppenmitgliedern, die den Go-Button schon gedrückt haben
\item[$\bullet$] Benachrichtigung über das 'Aufbrechen' anderer Gruppenmitglieder
\item[$\hookrightarrow$] Durch Drücken des Go-Buttons wird eine Benachrichtigung an den Server gesendet, der diese an die Mitglieder weiterleitet
\item[$\bullet$] Benachrichtigung bei ändern des Treffpunktes (Ort/Uhrzeit)
\item[$\hookrightarrow$] Gleiches Prinzip wie beim vorherigen Punkt
\item[$\bullet$] Treffpunkte für ein spezielles Datum festlegen
\end{description}
\subsection{Abgrenzungskriterien}
\begin{description}
\item[$\bullet$] Zuordnung der GPS-Daten zu einem Namen
\item[$\bullet$] Suchen von Gruppen oder Benutzern über die App
\item[$\hookrightarrow$] Der Server bietet keine Funktionen um Nutzer oder Gruppen zu finden
\item[$\bullet$] Beitreten der Gruppe ohne Einladungslink
\item[$\hookrightarrow$] Der Server bietet keine Möglichkeit einer Gruppe ohne Link beizutreten (auch wenn der Gruppenname bekannt ist!)
\item[$\bullet$] Abstimmung über Ort und Uhrzeit eines Treffpunktes\\bzw. gemeinsames vereinbaren
\item[$\bullet$] Angaben über Verspätungen zu einem vereinbarten Treffpunkt machen
\item[$\bullet$] Nachrichten, Fotos und anderer Dateien über die App versenden
\item[$\hookrightarrow$] Der Datenaustausch mit dem Server beschränkt sich auf die oben genannten Daten
\item[$\bullet$] Es werden nicht mehr als 32 Gruppen pro Benutzer unterstützt, d.h. der Benutzer kann nur in 32 Gruppen zur gleichen Zeit Administrator sein
\end{description}

