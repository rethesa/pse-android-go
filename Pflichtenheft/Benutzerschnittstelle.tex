\section{Benutzerschnittstelle}
\begin{wrapfigure} [H]
\caption{Erste Startansicht}
\includegraphics[scale = 0.5]{resources/images/startansicht.png}
\caption{
	[Bildüberschrift]Erste Startartansicht\\ \\

	\textbf{Beschreibung:}\\
	Erste Startansicht der App, Begrüßung des neuen Benutzers und Aufforderung an den Benutzer sich einen Benutzernamen zu wählen.\\
	\textbf{Elemente:}\\
	Textfeld "User name" zum Einfügen des Benutzernamens\\
	"Weiter"-Button um diesen zu bestätigen\\
	\textbf{Verwendung:}\\
	Durch einmaliges Tippen auf das Textfeld "User name" wird die Bildschirmtastatur aktiviert und der Benutzer kann seinen gewählten Benutzernamen eingeben.\\
	Durch einmaliges Tippen auf den "Weiter"-Button wird dieser Benutzername bestätigt und der Benutzer wird weiter geleitet zu der Gruppenübersicht\\ \\
}
\end{wrapfigure}

\begin{figure} [H]
	\caption{Gruppenübersicht}
\begin{center}
	\includegraphics[scale = 0.5]{resources/images/gruppenuebersicht.png}
\end{center}
\end{figure}

[Bildüberschrift]Gruppenübersicht\\ \\
\textbf{Beschreibung:}\\
Übersicht über die Gruppen, in denen der Benutzer Mitglied ist. Wenn der Benutzer das erste Mal zu dieser Ansicht gelangt, ist er in noch keiner Gruppe Mitglied und somit werden ihm auch noch keine angezeigt\\
\textbf{Elemente:}\\
Benutzernahme zur Orientierung, wie der Benutzer angemeldet ist\\
Gruppennamen zur Übersicht über die Gruppen, in denen der Benutzer Mitglied ist\\
"Gruppe hinzufügen"-Button um eine neue Gruppe hinzuzufügen\\
\textbf{Verwendung:}\\
WK(Wunschkriterium): Durch einmaliges Tippen auf den Benutzernamen wird der Benutzer weiter geleitet zu der Option Benutzername ändern\\
Durch einmaliges Tippen auf einen Gruppennamen wird der Benutzer weiter geleitet zu der Maps-Ansicht der Gruppe\\
Durch langes Tippen auf einen Gruppennamen kann der Benutzer "aus der Gruppe austreten" wählen bzw. der GA (Gruppenadministrator) "Gruppe löschen"\\
Durch einmaliges Tippen auf den "+Gruppe hinzufügen"-Button wird der Benutzer weiter geleitet zu der Option "Gruppe hinzufügen"\\
Durch Streichen nach oben oder unten kann der Benutzer durch die Gruppen scrollen, wenn es mehr sind als auf den Bildschirm passen\\
Durch Streichen von rechts nach links kann man die Gruppenansicht weg schieben und gelangt zu der zuletzt verwendeten Map\\ \\

\begin{figure} [H]
	\caption{Benutzername ändern (Wunschkriterium)}
\begin{center}
	\includegraphics[scale =0.5]{resources/images/username_aendern.png}
\end{center}
\end{figure}
[Bildüberschrift]Benutzername ändern (WK)\\ \\
\textbf{Beschreibung:}\\
Option zum ändern des BNs(Benutzernamen), Information über den aktuellen BN und Aufforderung an den Benutzer sich einen neuen BN zu wählen\\
\textbf{Elemente:}\\
BN zur Orientierung, wie der Benutzer angemeldet ist\\
Textfeld "User name" zum Einfügen des BNs\\
"Weiter"-Button um diesen zu bestätigen\\
\textbf{Verwendung:}\\
Durch einmaliges Tippen auf das Textfeld "User name" wird die Bildschirmtastatur aktiviert und der Benutzer kann seinen gewählten BN eingeben.\\
Durch einmaliges Tippen auf den "Weiter"-Button wird dieser BN bestätigt und der Benutzer wird weiter geleitet zu der Gruppenübersicht\\ \\

\begin{figure}[H]
	\caption{Gruppe erstellen}
\begin{center}
	\includegraphics[scale =0.5]{resources/images/gruppe_erstellen.png}
\end{center}
\end{figure}
[Bildüberschrift]Gruppe erstellen\\ \\
\textbf{Beschreibung:}\\
Option Gruppe hinzufügen, Aufforderung an den Benutzer sich einen Gruppennamen zu wählen\\
\textbf{Elemente:}\\
Textfeld "Gruppenname" zum Einfügen des Gruppennamens\\
"Weiter"-Button um diesen zu bestätigen\\
\textbf{Verwendung:}\\
Durch einmaliges Tippen auf das Textfeld "Gruppenname" wird die Bildschirmtastatur aktiviert und der Benutzer kann seinen gewählten Gruppennamen eingeben.\\
Durch einmaliges Tippen auf den "Weiter"-Button wird dieser Gruppenname überprüft. Ist dieser einmalig im System wird er bestätigt und der Benutzer wird weiter geleitet zu der Gruppenübersicht. Gibt es bereits eine Gruppe die den gewählten Namen trägt, wird der Benutzer aufgefordert einen anderen Gruppennamen auszuwählen.\\ \\


\begin{figure}[H]
	\caption{Map-Ansicht der Gruppe (leer)}
\begin{center}
	\includegraphics[scale =0.5]{resources/images/map_leer_Admin.png}
\end{center}
\end{figure}
[Bildüberschrift]Map-Ansicht der Gruppe (leer)\\ \\
\textbf{Beschreibung:}\\
Map-Ansicht der Gruppe. Wenn der GA(Gruppenadministrator) das erste Mal zu dieser Ansicht gelangt, hat er noch keine Treffen erstellt und somit wird ihm auch noch eine leere Map angezeigt. Ebenso wenn er kein aktuelles Treffen erstellt hat, wird allen GMG(Gruppenmitgliedern) eine leere Map angezeigt.\\
\textbf{Elemente:}\\
"Gruppenname"-Button um zu den Gruppendetails zu gelangen\\
"neues Treffen erstellen"-Button bei der Ansicht des GA um ein neues Treffen zu erstellen, bzw. "kein aktuelles Treffen"-Anzeige bei der Ansicht aller GMG ohne besondere Rechte\\
"Hier suchen"-Textfeld um einen Ort auf der Karte zu suchen\\
Lupen-Button um die Suche zu starten\\
Handle links unten in der Ecke um die Gruppenansicht wieder herauszuziehen\\
"Go"-Button, ausgegraut, inaktiv\\
\textbf{Verwendung:}\\
Durch einmaliges Tippen auf den "Gruppenname"-Button wird das GMG weiter geleitet zu der Ansicht "Gruppendetails ".\\
Durch einmaliges Tippen auf den "neues Treffen erstellen"-Button wird der GA weiter geleitet zu der Option "Treffen erstellen", für alle anderen GMG ist diese "kein aktuelles Treffen"-Anzeige lediglich informativ.\\
Durch einmaliges Tippen auf das Textfeld "Hier suchen" wird die Bildschirmtastatur aktiviert und das GMG kann seinen gewählten Ort eingeben.\\
Durch einmaliges Tippen auf den Lupen-Button wird eine Suche nach dem gewählten Ort gestartet und die Ergebnisse dem GMG angezeigt\\
Durch Streichen von links nach rechts über den Hanlde-Button kann das GMG die Gruppenansicht wieder herausziehen\\ \\

\begin{figure} [H]
	\caption{Übersicht über die Gruppenmitglieder}
\begin{center}
	\includegraphics[scale =0.5]{resources/images/gruppendetails_Admin.png}
\end{center}
\end{figure}
[Bildüberschrift]Übersicht über die Gruppenmitglieder\\ \\
\textbf{Beschreibung:}\\
Übersicht über die GMGer. Wenn der GA das erste Mal zu dieser Ansicht gelangt, ist außer ihm noch kein weiteres Mitglied in dieser Gruppe und somit werden ihm auch noch keine weiteren angezeigt\\
\textbf{Elemente:}\\
"Gruppenname"-Button zur Übersicht, in welcher Gruppe das GMG aktiv ist und um zurück zur Map-Ansicht der Gruppe zu gelangen\\
"neues Treffen erstellen"-Button oder Uhrzeit-Button bei der Ansicht des GA um ein neues Treffen zu erstellen, bzw. "kein aktuelles Treffen"-Anzeige oder Uhrzeit-Anzeige bei der Ansicht aller GMG ohne besondere Rechte\\
Mitgliedernamen zur Übersicht über die Mitglied der Gruppe\\
"Mitglied hinzufügen"-Button um ein neues Mitglied einzuladen bei der Ansicht des GA\\
\textbf{Verwendung:}\\
Durch einmaliges Tippen auf den "Gruppenname"-Button wird der Gruppenadministrator zurück geleitet zu der Map-Ansicht der Gruppe\\
"neues Treffen erstellen"-Button bzw. Uhrzeit-Button hat die selbe Funktion wie in der Map-Ansicht der Gruppe\\
Durch einmaliges Tippen auf den "+Mitglied hinzufügen"-Button öffnet sich das Fenster "Link versenden"\\ \\

\begin{figure} [H]
	\caption{Link versenden}
\begin{center}
	\includegraphics[scale =0.5]{resources/images/link_versenden.png}
\end{center}
\end{figure}
[Bildüberschrift]Link versenden\\ \\
\textbf{Beschreibung:}\\
Übersicht über verschiedene Kommunikationswege, die verwendet werden können um andere Mitglieder in die Gruppe einzuladen.\\
\textbf{Elemente:}\\
beliebig viele Button der Kommunikationswegen, die der GA auf seinem Android-Gerät installiert hat\\
\textbf{Verwendung:}\\
Durch einmaliges Tippen auf einen der Button wird der GA weiter geleitet zu der ausgewählten Applikation über die er dann den neu generierten Gruppen-Einladungs-Link versenden kann\\ \\

\begin{figure} [H]
	\caption{Treffen erstellen}
\begin{center}
	\includegraphics[scale =0.5]{resources/images/treffpunkt_erstellen.png}
\end{center}
\end{figure}
[Bildüberschrift]Treffen erstellen \\ \\
\textbf{Beschreibung:}\\
Option zum Erstellen eines neuen Treffens, Aufforderung an den Gruppenadministrator einen Zielort und eine Uhrzeit zu wählen.\\
\textbf{Elemente:}\\
Textfeld "Zielort" zum Einfügen des Zielortes\\
Map-Symbol-Button zum Auswählen des Zielortes per Map\\
Textfeld "Uhrzeit" zum Einfügen der Uhrzeit\\
"Weiter"-Button um diese zu bestätigen\\
\textbf{Verwendung:}\\
Durch einmaliges Tippen auf das Textfeld "Zielort" wird die Bildschirmtastatur aktiviert und der GA kann seinen gewählten Zielort eingeben.\\
Alternativ: durch einmaliges Tippen auf den Map-Symbol-Button wird der GA weiter geleitet zu der Map-Ansicht und kann dort durch einmaliges Tippen auf einen Ort diesen als Zielort eingeben\\
Durch einmaliges Tippen auf das Textfeld "Uhrzeit" wird die Bildschirm-Zahlentastatur aktiviert und der GA kann seine gewählte Zielzeit eingeben.\\
Durch einmaliges Tippen auf den "Weiter"-Button werden dieser Zielort und diese Zielzeit überprüft. Gibt es den Zielort und diese Zielzeit, werden diese bestätigt und der Gruppenadministrator wird weiter geleitet zu der Map-Ansicht der Gruppe. Gibt es diesen Ort oder diese Zeit nicht, wird der GA aufgefordert einen anderen Ort, bzw eine andere Zeit auszuwählen\\

\begin{figure}[H]
	\caption{Map-Ansicht der Gruppe}
\begin{center}
	\includegraphics[scale =0.5]{resources/images/map.png}
\end{center}
\end{figure}
[Bildüberschrift]Map-Ansicht der Gruppe\\ \\
\textbf{Beschreibung:}\\
Map-Ansicht der Gruppe mit festgelegtem Treffpunkt. Allen GMG wird in dieser Ansicht nicht nur die Map angezeigt, sondern auch der Zielort und die Uhrzeit des nächsten festgelegten Treffpunktes, außerdem noch auf der Map der Zielort durch eine Stecknadel beschriftet mit "Z".\\
\textbf{Elemente:}\\
"Gruppenname"-Button um zu den Gruppendetails zu gelangen\\
"Zielort - Uhrzeit"-Ansicht zur Orientierung wann und wo das nächste Treffen festgelegt ist, bzw. "Zielort - Uhrzeit"-Button bei der Ansicht des GA um ein neues Treffen zu erstellen\\
"Hier suchen"-Textfeld um einen Ort auf der Karte zu suchen\\
Lupen-Button um die Suche zu starten\\
Handle links unten in der Ecke um die Gruppenansicht wieder herauszuziehen\\
Aktiver "Go"-Button um anzuzeigen, wann man los geht
\textbf{Verwendung:}\\
Durch einmaliges Tippen auf den "Gruppenname"-Button wird das GM weiter geleitet zu der Ansicht "Gruppendeteils".\\
Durch einmaliges Tippen auf den "Zielort - Uhrzeit"-Button wird der GA weiter geleitet zu der Option "Treffen erstellen"\\
Durch einmaliges Tippen auf das Textfeld "Hier suchen" wird die Bildschirmtastatur aktiviert und das GMG kann seinen gewählten Ort eingeben.\\
Durch einmaliges Tippen auf den Lupen-Button wird eine Suche nach dem gewählten Ort gestartet und die Ergebnisse dem GMG angezeigt\\
Durch Streichen von links nach rechts über den Hanlde-Button kann das GMG die Gruppenansicht wieder herausziehen\\
Durch einmaliges Tippen auf den "Go"-Button sendet das GMG seinen Standort in regelmäßigen Zeitabständen zu den anderen GMG und erhält auch alle nötigen Informationen über diese (Ansicht: "Map-Ansicht der Gruppe - treffen")\\ \\

\begin{figure}
	\caption{Map-Ansicht der Gruppe - treffen}
\begin{center}
	\includegraphics[scale =0.5]{resources/images/map_Go.png}
\end{center}
\end{figure}
[Bildüberschrift]Map-Ansicht der Gruppe - treffen\\ \\
\textbf{Beschreibung:}\\
Map-Ansicht der Gruppe wenn der "Go"-Button schon gedrückt ist. Allen GM wird in dieser Ansicht nicht nur die Map angezeigt, sondern auch der Zielort und die Uhrzeit des nächsten festgelegten Treffpunktes, außerdem noch auf der Map der Zielort durch eine Stecknadel beschriftet mit "Z" und die Standorte der anderen GMer durch blaue Kreise. Je größer ein Kreis, desto mehr GMG befinden sich an dem gleichen Standort.\\
\textbf{Elemente:}\\
"Gruppenname"-Button um zu den Gruppendetails zu gelangen\\
"Zielort - Uhrzeit"-Ansicht zur Orientierung wann und wo das nächste Treffen festgelegt ist, bzw. "Zielort - Uhrzeit"-Button bei der Ansicht des GA um ein neues Treffen zu erstellen\\
"Hier suchen"-Textfeld um einen Ort auf der Karte zu suchen\\
Lupen-Button um die Suche zu starten\\
Handle links unten in der Ecke um die Gruppenansicht wieder herauszuziehen\\
Deaktiver "Go"-Button um das Versenden seines Standortes zu stoppen\\
Blaue Kreise zur Orientierung, wo sich die anderen GM befinden\\
\textbf{Verwendung:}\\
Durch einmaliges Tippen auf den "Gruppenname"-Button wird das GMG weiter geleitet zu der Ansicht "Gruppendeteils".\\
Durch einmaliges Tippen auf den "Zielort - Uhrzeit"-Button wird der GA weiter geleitet zu der Option "Treffen erstellen"\\
Durch einmaliges Tippen auf das Textfeld "Hier suchen" wird die Bildschirmtastatur aktiviert und das GMG kann seinen gewählten Ort eingeben.\\
Durch einmaliges Tippen auf den Lupen-Button wird eine Suche nach dem gewählten Ort gestartet und die Ergebnisse dem GMG angezeigt\\
Durch Streichen von links nach rechts über den Hanlde-Button kann das GMG die Gruppenansicht wieder herausziehen\\
Durch einmaliges Tippen auf den "Go"-Button wird dieser wieder aktiv und das Senden des Standortes des GMG wird gestoppt. Außerdem kann das GMG dann nicht mehr die Standorte der anderen GMG sehen (Ansicht: "Map-Ansicht der Gruppe")\\ \\
