\newpage
\section{Musskriterien}
Die Musskriterien sind in Client- und Serverseitige Interaktionen aufgeteilt.\\
\textbf{/FCXX/} steht hierbei für clientseitige Aktionen.\\
\textbf{/FSXX/} steht für die serverseitigen Reaktionen.\\
Die Nummer ist bei zusammenhängenden Aktionen gleich.\\
\subsection{Clientseitig}
     \textbf{/FC010/} Registrierung von Benutzern durch einmalige Namensfestlegung\\
     \textbf{/FC020/} Löschen des eigenen Nutzeraccounts\\
     \textbf{/FC030/} Erstellen von Gruppen mit eindeutigem Namen. Benutzer der die Gruppe erstellt wird zum Gruppenadministrator\\
     \textbf{/FC040/} Löschen von Gruppen (nur der Gruppenadministrator)   \\
     \textbf{/FC050/} Erstellen von Gruppenlinks. Diese werden als Einladung an zukünftige Mitglieder über externe Messenger versendet (nur Admin)\\
     \textbf{/FC060/} Beitreten einer Gruppe nur über Einladungslink (Voraussetzung registriert) \\
     Der Link wird über die App geöffnet (Falls installiert) \\
     \textbf{/FC070/} Verlassen einer Gruppe (als Gruppenmitglied)\\
     \textbf{/FC080/} Entfernen von Gruppenmitgliedern (nur Gruppenadministrator): entspricht verlassen einer Gruppe, nur erzwungen durch Gruppenadministrator\\
     \textbf{/FC090/} Festlegen von Treffpunkten: einstellen von Zielort und Uhrzeit (nur Gruppenadministrator)\\
     \textbf{/FC100/} Go-Button: durch drücken des Go-Buttons eigene Position an die Gruppe übermitteln (alle Gruppenmitglieder)\\
     \textbf{/FC110/} Anzeigen der gemittelten GPS-Daten der Gruppe auf der Karte; sind mehrere Mitglieder innerhalbt eines bestimmten Radius,
     werden sie als Gruppe angezeigt\\
     \textbf{/FC120/} Der Datenverkehr von Client zu Server muss verschlüsselt sein\\
\subsection{Serverseitig}
     \textbf{/FS010/} Name auf eindeutige ID abbilden und (Name,ID) an Server senden\\
     \textbf{/FS020/} App auf anfangszustand zurücksetzen, Nutzer aus allen Gruppen entfernen und das Tupel (Name, ID) vom Server löschen \\
     \textbf{/FS030/} Auf dem Server wird eine neue Gruppe angelegt mit Gruppenadministrator als einzigem Mitglied\\
     \textbf{/FS040/} Entferne alle Mitglieder aus der Gruppe und Lösche Gruppe vom Server\\
     \textbf{/FS050/} Gruppenlink wird zum Server gesendet und in Gruppe gespeichert. Nach der Nutzung wird er gelöscht\\
     \textbf{/FS060/} Link wird an Server gesendet und die zugehörige Gruppe identifiziert.\\ Die ID des Benutzers wird dann der Gruppenliste hinzugefügt\\
     \textbf{/FS070/} Die ID des Benutzers wird aus der Gruppenliste auf dem Server gelöscht. Die aktuelle Liste wird mit den verbleibenden Benutzern synchronisiert\\
     \textbf{/FS090/} Parameter (Uhrzeit/Zielort) wird an Server gesendet und in der Gruppe gespeichert. Dann wird der Parameter mit den Mitgliedern synchronisiert \\
     \textbf{/FS100/} Position als GPS Koordinaten an Server senden, der sie dann an die Mitglieder verteilt\\
     \textbf{/FS110/} Daten der einzelnen Mitglieder werden über den Server an die Gruppe verteilt \\
     \textbf{/FS120/} Der Datenverkehr von Server zu Client muss verschlüsselt sein\\

\section{Wunschkriterien}
\subsection{Clientseitig}
     \textbf{/FC130/} Aktivieren und Deaktivieren des GPS-Tracking unabhängig vom Go-Button\\
     \textbf{/FC140/} Änderung des Benutzernamens nach erstmaliger Registrierung\\
     \textbf{/FC150/} Gruppenmitglieder zu Administratoren machen                \\
     \textbf{/FC160/} Änderung des Gruppennamens nach erstellen der Gruppe        \\
     \textbf{/FC170/} Benutzer nach langer Inaktivität löschen                     \\
     \textbf{/FC180/} Anzeigen der Namen von Gruppenmitgliedern, die den Go-Button schon gedrückt haben\\
     \textbf{/FC190/} Benachrichtigung über das 'Aufbrechen' anderer Gruppenmitglieder\\
     \textbf{/FC200/} Benachrichtigung bei ändern des Treffpunktes (Ort/Uhrzeit)\\
     \textbf{/FC210/} Treffpunkte für ein spezielles Datum festlegen\\
\subsection{Serverseitig}
     \textbf{/FS140/} Neuer Name muss mit dem Server synchronisiert werden bzw. mit allen Gruppen in denen der Benutzer ist\\
     \textbf{/FS150/} Dem Server mitteilen, dass bei gewünschtem Mitglied das Admin-Flag gesetzt werden soll.\\
     \textbf{/FS160/} Neuer Name muss erst mit dem Server, dann mit allen Mitgliedern synchronisiert werden\\
     \textbf{/FS170/} Der Server speichert den Zeitpunkt der letzten Aktivität von jedem Benutzer\\
     \textbf{/FS190/} Durch Drücken des Go-Buttons wird eine Benachrichtigung an den Server gesendet, der diese an die Mitglieder weiterleitet\\
     \textbf{/FS200/} Gleiches Prinzip wie beim vorherigen Punkt\\
