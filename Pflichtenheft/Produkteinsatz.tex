
\section{Produkteinsatz}
Das Produkt dient zur erleichterten Vermittlung zwischen Gruppenmitgliedern, um sich an einem bestimmten Zielort zu einer festgelegten Uhrzeit zu treffen. Dabei steht im Vordergrund sich gemeinsam von einer unweit entfernten Startposition auf den Weg zu machen und den Rest der Gruppe zu finden. 
Durch die anonyme Angabe der aktuellen Standorte aller Gruppenmitglieder auf der Karte, kann man so entweder auf andere warten oder diese einholen und gemeinsam den Zielort erreichen. Halten sich mehrere Mitglieder innerhalb eines Radius von 25 Metern auf, werden die Standorte dieser Mitglieder gemittelt und in Relation größer auf der Karte angezeigt, als nur der Standort eines einzelnen Mitglieds.\\

\subsection{Anwendungsbereiche}
\textbf{Privater Anwendungsbereich} \\
Im privaten Gebrauch kann das Produkt zur Verabredung zum gemeinsamen Essen oder zu anderen Freizeitaktivitäten verwendet werden. So hat jedes Mitglied Einsicht darüber, ob sich die anderen Gruppenmitglieder schon auf den Weg gemacht haben, sich verspäten werden oder wie lange sie noch brauchen um den Zielort zu erreichen.\\

\textbf{Kommerzieller Anwendungsbereich} \\
Im kommerziellen Bereich kann das Produkt seine Anwendung in der Planung von Konferenzen finden. So erfahren alle Gruppenmitglieder wo und wann die nächste Besprechung stattfindet. \\

\subsection{Zielgruppen}
Zielgruppen sind sowohl Privatpersonen als auch Unternehmen die sich in Gruppen organisieren und nicht jeden Einzelnen über einen neu anstehenden Treffpunkt informieren möchten. Bei nahe liegenden Startpositionen kommt noch hinzu, dass man sich anderen Mitgliedern anschließen und gemeinsam den Weg beschreiten kann, da deren Standorte auf der Karte sichtbar sind.\\

\subsection{Betriebsbedingungen}
Das Produkt kann bei einer bestehenden Internet- und GPS- Verbindung genutzt werden. Die GPS Technologie ist nötig um die Standorte aller Gruppenmitglieder zu erfassen und mit ausreichender Internetverbindung an die anderen Mitglieder zu übertragen. Bei Fehlen einer der beiden Komponenten ist die Funktionalität des Produkts nicht gewährleistet.\\

