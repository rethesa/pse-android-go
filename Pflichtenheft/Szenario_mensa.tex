Der Professor erstellt eine Gruppe und wird zum Gruppenadministrator. Danach lädt der Prof die Mitarbeiter, mit denen er gerne zu Mittag essen möchte, per Link ein. Der Prof hält bis 13:00 Uhr eine Vorlesung und möchte danach in der Mensa essen gehen. Also setzt er den Zeitpunkt zum essen, auf 13:10 Uhr und den Standort auf „Mensa Adenauerring“. Seine eingeladenen Kollegen und Mitarbeiter treten der Gruppe per Link bei. Es ist 13:00 Uhr, der Prof drückt den „Go-Button“, packt seine Sachen und geht los zur Mensa. Nach und nach schauen die Personen in der Gruppe in die App rein und sehen, dass Prof unterwegs zur Mensa ist. Fünf der Mitarbeiter des Profs, machen sich sofort auf den Weg und drücken ebenfalls den „Go-Button“. Jedoch ist ein Mitarbeiter in einem Gespräch verwickelt, und bemerkt das Treffen nicht. Um ca. 13:10 treffen sich der Prof und die ersten fünf Mitarbeiter an der Mensa, dort beschließen sie gemeinsam essen zu gehen. Nach etwa 5 Minuten hat der fehlende Mitarbeiter sein Gespräch beendet, dann schaut er in die App und sieht, dass sich seine Kollegen und der Professor bereits in der Mensa befinden. Er bemerkt ebenfalls, dass das Treffen vor 5 Minuten los ging und beschließt nach zu kommen. Der Mitarbeiter drückt dann auch auf den „Go-Button“ und hastet zur Mensa. Dort angekommen trifft sich der fehlende Mitarbeiter mit der Gruppe in der Schlange zur Linie 1. Ein weiterer Mitarbeiter, war zuvor sehr vertieft in seiner Arbeit und nimmt deshalb die Einladung zu spät an. Dieser nimmt jedoch die Einladung zu spät an. Er drückt dennoch auf den „Go-Button“ und macht sich auf den Weg zur Mensa. Die Gruppe sieht in der App, dass noch ein Nachzügler kommt und wartet auf den verlorenen Mitarbeiter und Kollegen.\\