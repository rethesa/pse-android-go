\section{Qualitätsbestimmung}
\textbf{Produktqualität:}\\ %-- 2. wichtigste
% Funktionalität - Angemessenheit - Richtigkeit - Interoperabilität - Ordnungmäßigkeit - Sicherheit\\
Die Qualität der App ist auf einem hohen Niveau. Vor allem im Bezug auf die Datensicherheit sind die Daten der Benutzer geschützt.\\
\textbf{Zuverlässigkeit:}\\% -- sehr wichtig
% Fehlertoleranz - Wiederherstellbarkeit\\
Zuverlässigkeit ist unser wichtigstes Qualitätsziel. Unsere App soll unter Durchschnittsanfragen stabil laufen und eine große Fehlertoleranz haben. \\
\textbf{Benutzbarkeit:}\\ %-- 3. wichtigste
%Verständlichkeit - Erlernbarkeit - Bedienbarkeit\\
Die App ist übersichtlich und einfach zu bedienen. Dadurch sollen die Lernkurve gering gehalten werden, damit auch Kunden angesprochen werden, die wenige bis kaum Apps benutzen.\\
\textbf{Effizienz:}\\ %-- nicht sooo wichtig
%Zeitverhalten - Verbrauchsverhalten\\
\\
\textbf{Änderbarkeit:}\\ %-- weniger wichtig
%Analysierbarkeit - Modifizierbarkeit - Stabilität - Prüfbarkeit\\
\\
\textbf{Übertragbarkeit:}\\% -- weniger wichtig
%Anpassbarkeit - Installierbarkeit - Austauschbarkeit\\
Der Programm Code ist Erweiterbar.\\

KOMMENTAR:
was macht die Qualität der App aus? Der Satz steht so mitten im Raum, als suche er Freunde
Der 2. Satz ist sinnvoll. Aber doppelt gemoppelt: Datensicherheit... Daten geschützt
ebenso die Zuverlässigkeit... willst du nicht schreiben, was zuverlässig sein soll?
nicht noch rein bringen, dass sie auch auf bekannten apps aufbaut? vielleicht so: "Die App ist übersichtlich und ähnlich wie bereits bekanntere Apps zu bedienen. Trotzdem ist die Bedienung auch für Kunden, die bisher kaum mit Android-Geräten gearbeitet haben, leicht ersichtlich. Dadurch wird die Lernkurve (aber was für eine Lernkurve überhaupt) gering gehalten." (vermeide bandwurmsätze ;) )

was macht diese einsame "Änderbarkeit" ohne Unterpunkte?