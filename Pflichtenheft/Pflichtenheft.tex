\documentclass[parskip=full]{scrartcl}
\usepackage[utf8]{inputenc}
\usepackage[T1]{fontenc}
\usepackage[german]{babel}
\usepackage{hyperref}
\hypersetup{ 
pdftitle={
PSE: Pflichtenheft},
bookmarks=true,
}
\usepackage{csquotes}



\begin{document}
\title{Android Go-App}
\author{Tarek, \\Dennis, \\Matthias, \\Victoria Karl, \\Theresa Heine\\
	\\Betreuer: \\Heiko Klare, \\ Erik ...\\}	
\date{\today}
\maketitle
\newpage
\tableofcontents
\newpage

\section{Einleitung}

\section{Zielbestimmung}
\subsection{Musskriterien}

\subsection{Wunschkriterien}
\subsection{Abgrenzungskriterien}

\section{Produkteinsatz}
\subsection{Anwendungsbereiche}
\subsection{Zielgruppen}
\subsection{Betriebsbedingungen}

\section{Produktumgebung}
\subsection{Softwareanforderungen}
\subsection{Hardwareanforderungen}

\section{Andwendungsfälle/ Produktübersicht}
\subsection{Musskriterien}
\subsection{Wunschkriterien}



\section{Produktfunktionen/ Systemmodelle}
Funktionsübersicht: \\
F10 Erstmaliges Öffnen der App (Registrierung) Tarek \\
F20 Gruppe erstellen Dennis \\
F30 Mitglieder einladen Theresa \\
F40 Gruppe beitreten Dennis \\
F50 Treffpunkt (Ort/Zeit) festlegen (Admin) Vicky \\
F60 Gemeinsames Losgehen Matthias \\
F70 Gruppe verlassen Vicky \\
F80 Gruppe löschen Tarek \\
\subsection{Funktionale Anforderungen}
aufgeteilt in Serverseitige und Nutzerseitige (und in Muss- und Wunschkriterien)
\subsection{Nichtfunktionale Anforderungen}

\section{Produktdaten}
\subsection{Personendaten}
\subsection{Serverdaten?}

\section{Produktleistungen/ Qualitätsziele}

\section{Benutzerschnittstelle}

\section{Globale Testfälle}
\subsection{Testfälle für Musskriterien}
\subsection{Testfälle für Wunschkriterien}

\section{Qualitätsbestimmung}
Funktionalität 
Zuverlässigkeit
Benutzbarkeit 
Effizienz 
Änderbarkeit 
Übertragbarkeit

\section{Entwicklungsumgebung}

\section{A.Anhang}
Bilder und Flipcharts

\section{B.Anhang}
Glossar


\end{document}