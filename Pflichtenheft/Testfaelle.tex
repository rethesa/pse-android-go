\section{Testfälle}
\subsection{Testfälle für Musskriterien}
%\begin{tabular}{ll}
\textbf{/T010/} Benutzer registrieren (FC010) \\
\begin{itemize}
Der Benutzer wählt einen Benutzernamen.
ist der Benutzername gültig, liegt auf dem Server ein Tupel (Benutzername, ID) vor.
\end{itemize}

\textbf{/T020/} Benutzeraccount löschen (FC020) \\
\begin{itemize}
Der Benutzer wählt aus, seinen Account zu löschen.
Bei erfolgreicher Ausführung ist das Tupel (Benutzername, ID) vom Server gelöscht.
\end{itemize}

\textbf{/T030/} Erstellen einer Gruppe (FC030)\\
\begin{itemize}
Der Benutzer wählt Gruppe erstellen im Menü aus und gibt einen Namen ein.
Bei eindeutigem Namen liegt auf dem Server eine Gruppe vor, mit dem Ersteller als einzigem Mitglied.
\end{itemize}

\textbf{/T040/} Löschen einer Gruppe (FC040)\\
Der Benutzer wählt aus, eine seiner selbst erstellten Gruppen zu löschen.
Bei Erfolg ist die Gruppe sowohl vom Server, als auch bei den Mitgliedern gelöscht.
\begin{itemize}

\textbf{/T050/} Erstellen eines Gruppenlinks (FC050)\\
Der Benutzer hat eine Gruppe erstellt und will Mitglieder einladen.
Er wählt im Gruppenmenü der entsprechenden Gruppe "Mitglieder einladen" aus.
Bei Erfolg erhält der Benutzer einen Gruppenlink, der auch in der Gruppe gespeichert ist.
\begin{itemize}

\textbf{/T060/} Beitreten einer Gruppe (FC060)\\
\begin{itemize}
Voraussetzung: Gruppe besteht, Benutzer ist registriert.
Der Benutzer bekommt einen Link zugesendet und öffnet ihn mit der App.
Bei erfolgreichem Beitritt, ist der Benutzer in der Liste der entsprechenden Gruppe.
Außerdem ist der genutzte Link aus der Gruppe gelöscht.
\end{itemize}

\textbf{/T070/} Verlassen einer Gruppe (FC070)\\
\begin{itemize}
Voraussetzung: Benutzer ist mitglied in der Gruppe.
Benutzer wählt "Gruppe verlassen" aus.
Bei Erfolg ist der Benutzer nicht mehr in der Gruppenliste auf dem Server.
Zudem ist die Gruppe nicht mehr in der Liste des Benutzers.
\end{itemize}

\textbf{/T080/} Entfernung von Gruppenmitgliedern (FC080)\\
\begin{itemize}
Voraussetzung: Benutzer ist Admin in der Gruppe.
Der Benutzer wählt bei einem Mitglied "Mitglied entfernen" aus.
Bei erfolgreicher Ausführung ist das Mitglied nicht mehr in seiner Liste.
Auf dem Server ist das Mitglied aus der Gruppenliste gelöscht.
\end{itemize}

\textbf{/T090/} Festlegen des Zielorts (FC090)\\
\begin{itemize}
Voraussetzung: Benutzer ist Admin in der Gruppe.
Der Benutzer editiert den Zielort der Gruppe.
Bei Erfolg ist der neue Zielort in der Gruppe sowohl beim Benutzer,
als auch auf dem Server gespeichert.
\end{itemize}

\textbf{/T100/} Festlegen der Uhrzeit (FC100)\\
\begin{itemize}
Voraussetzung: Benutzer ist Admin in der Gruppe.
Der Benutzer editiert die Uhrzeit der Gruppe.
Bei Erfolg ist die neue Uhrzeit beim Benutzer und auf dem Server gespeichert.
\end{itemize}




\textbf{/T040/} Benutzer löschen \\
\begin{itemize}
\setlength{\itemsep}{0pt}
\item Der Benutzer ist registriert. Dies wird über die auf dem Gerät
und Server gespeicherten Daten geprüft.
\item Der Benutzer löscht seinen Account. Dies wird in der Datenbank des Servers geprüft.
\item Im Fehlerfall wird dem Benutzer eine Fehlermeldung angezeigt.
\end{itemize}

\textbf{/T030/} Gruppe erstellen \\
\begin{itemize}
\setlength{\itemsep}{0pt}
\item Der Benutzer ist registriert. Dies wird über die auf dem Gerät
und Server gespeicherten Daten geprüft.
\item Der Benutzer erstellt eine Gruppe. () Dies wird in der Datenbank des Server geprüft.
\item Im Fehlerfall wird dem Benutzer eine Fehlermeldung angezeigt.
\end{itemize}

\textbf{/T040/} Gruppe löschen \\
\begin{itemize}
\setlength{\itemsep}{0pt}
\item Der Benutzer ist Administrator der Gruppe. Dies wird in der Datenbank des Servers geprüft.
\item Der Benutzer löscht die Gruppe. () Dies wird in der Datenbank des Servers geprüft.
\item Im Fehlerfall wird dem Benutzer eine Fehlermeldung angezeigt.
\end{itemize}

\textbf{/T050/} Einladung erstellen \\
\begin{itemize}
\setlength{\itemsep}{0pt}
\item Der Benutzer ist Administrator der Gruppe. Dies wird in der Datenbank des Servers geprüft.
\item Der Benutzer erstellt () und verschickt Einladungen. Dies wird in der Datenbank des Servers geprüft.
\item Im Fehlerfall wird dem Benutzer eine Fehlermeldung angezeigt.
\end{itemize}


\textbf{/T060/} Gruppe beitreten \\
\begin{itemize}
\setlength{\itemsep}{0pt}
\item Der Benutzer wurde von einem Administrator zu einer Gruppe eingeladen. ()
\item Der Benutzer tritt der Gruppe () bei. Dies wird in der Datenbank des Servers geprüft.
\item Im Fehlerfall wird dem Benutzer eine Fehlermeldung angezeigt.
\end{itemize}


\textbf{/T070/} Gruppe verlassen \\
\begin{itemize}
\setlength{\itemsep}{0pt}
\item Der Benutzer ist Mitglied der Gruppe. Dies wird in der Datenbank des Servers geprüft.
\item Der Benutzer verlässt die Gruppe. () Dies wird in der Datenbank des Servers geprüft.
\item Im Fehlerfall wird dem Benutzer eine Fehlermeldung angezeigt.
\end{itemize}


\textbf{/T080/} Aus Gruppe entfernen \\
\begin{itemize}
\setlength{\itemsep}{0pt}
\item Der Benutzer ist Administrator der Gruppe. Dies wird in der Datenbank des Servers geprüft.
\item Der Benutzer wählt ein Mitglied zum Entfernen aus. Dies wird in der Datenbank des Servers geprüft.
\item Im Fehlerfall wird dem Benutzer eine Fehlermeldung angezeigt.
\end{itemize}


\textbf{/T090/} Treffpunkt festlegen \\
\begin{itemize}
\setlength{\itemsep}{0pt}
\item Der Benutzer ist Administrator der Gruppe. Dies wird in der Datenbank des Servers geprüft.
\item Der Benutzer legt einen Treffpunkt fest. () Dies wird in der Datenbank des Servers geprüft.
\item Im Fehlerfall wird dem Benutzer eine Fehlermeldung angezeigt.
\end{itemize}


\textbf{/T100/} Go-Button drücken / Aufbrechen \\ % TODO Formulierung?
\begin{itemize}
\setlength{\itemsep}{0pt}
\item Der Benutzer ist Mitglied einer Gruppe. Dies wird in der Datenbank des Servers geprüft.
\item Ein Administrator hat einen Treffpunkt festgelegt. () Dies wird in der Datenbank des Servers geprüft.
\item Der Benutzer bricht auf / drückt den Go-Button. ()
Seine Position wird übermittelt.
Dies wird beim Server geprüft.
\item Im Fehlerfall wird dem Benutzer eine Fehlermeldung angezeigt.
\end{itemize}


\textbf{/T110/} Positionen von Gruppenmitgliedern anzeigen \\
\begin{itemize}
\setlength{\itemsep}{0pt}
\item Der Benutzer ist Mitglied einer Gruppe. Dies wird in der Datenbank des Servers geprüft.
\item Mindestens ein Mitglied ist aufgebrochen. () Dies wird in der Datenbank des Servers geprüft.
\item Der Benutzer ruft die Karte auf und sieht die Positionen der
aufgebrochenen Mitglieder. () Dies wird auf dem Gerät überprüft.
\item Im Fehlerfall wird dem Benutzer eine Fehlermeldung angezeigt.
\end{itemize}


%\end{tabular}

\subsection{Testfälle für Wunschkriterien}
\begin{tabular}{ll}
/T200/ & Aufbrechen mit aktivierten/deaktiviertem GPS-Tracking \\ % TODO Formulierung?
/T210/ & Änderung des Benutzernamens \\
/T220/ & Gruppenmitglied als zusätzlichen Administrator setzen \\
/T230/ & Änderung des Gruppennamens \\
/T240/ & Inaktive Gruppen löschen \\
/T250/ & Inaktive Benutzer löschen \\
/T260/ & Anzeigen des Aufbruchs-Status von Gruppenmitgliedern \\
/T270/ & Benachrichtigung über Aufbrechen von Gruppenmitgliedern \\
/T280/ & Benachrichtigung über Änderung des Treffpunktes \\
/T290/ & Festlegen von Treffpunkten für spezifische Zeitpunkte \\
\end{tabular}
