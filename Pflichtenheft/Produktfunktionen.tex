\newpage
\section{Musskriterien}
Die Musskriterien sind in Client- und Serverseitige Interaktionen aufgeteilt.\\
\textbf{/FCXX/} steht hierbei für clientseitige Aktionen.\\
\textbf{/FSXX/} steht für die serverseitigen Reaktionen.\\
Die Nummer ist bei zusammenhängenden Aktionen gleich.\\
\subsection{Clientseitig}
     \textbf{/FC010/} Registrierung von Benutzern durch einmalige Namensfestlegung\\
     \textbf{/FC020/} Löschen des eigenen Nutzeraccounts\\
     \textbf{/FC030/} Erstellen von Gruppen mit eindeutigem Namen. Benutzer der die Gruppe erstellt wird zum Gruppenadministrator\\
     \textbf{/FC040/} Löschen von Gruppen (nur der Gruppenadministrator)   \\
     \textbf{/FC050/} Erstellen von Gruppenlinks (nur Admin). Diese dienen als Einladung für andere Benutzer \\
     \textbf{/FC060/} Beitreten einer Gruppe nur über Einladungslink (Voraussetzung registriert) \\
     Der Link wird über die App geöffnet (Falls installiert) \\
     \textbf{/FC070/} Verlassen einer Gruppe (als Gruppenmitglied)\\
     \textbf{/FC080/} Entfernen von Gruppenmitgliedern (nur Gruppenadministrator): entspricht verlassen einer Gruppe, nur erzwungen durch Gruppenadministrator\\
     \textbf{/FC090/} Festlegen von Zielort: einstellen von Zielort für ein Treffen (nur Gruppenadministrator)\\
     \textbf{/FC100/} Festlegen von Uhrzeit: einstellen der Uhrzeit für ein Treffen (nur Gruppenadministrator) \\
     \textbf{/FC110/} Go-Button: durch drücken des Go-Buttons eigene Position an die Gruppe übermitteln (alle Gruppenmitglieder)\\
     \textbf{/FC120/} Anzeigen der gemittelten GPS-Daten der Gruppe auf der Karte; sind mehrere Mitglieder innerhalbt eines bestimmten Radius,
     werden sie als Gruppe angezeigt\\
     \textbf{/FC130/} Der Datenverkehr von Client zu Server wird verschlüsselt\\
     \textbf{/FC140/} Go-Button: durch erneutes Drücken des Go-Buttons Positionsfreigabe deaktivieren\\
     \textbf{/FC150/} Abrufen der aktuellen Gruppenparameter vom Server bei geöffneter App\\
\subsection{Serverseitig}
     \textbf{/FS010/} Benutzername auf eindeutige ID abbilden und (Name,ID) speichern\\
     \textbf{/FS020/} Nutzer aus allen Gruppen entfernen und das Tupel (Name, ID) vom Server löschen \\
     \textbf{/FS030/} Eine neue Gruppe anlegen Gruppenadministrator als einzigem Mitglied\\
     \textbf{/FS040/} Entferne alle Mitglieder aus der Gruppe und Lösche Gruppe vom Server\\
     \textbf{/FS050/} Erstellten Gruppenlink in Gruppe speichern. Nach der Nutzung wird er gelöscht\\
     \textbf{/FS060/} Link empfangen und zugehörige Gruppe identifizieren.\\ Die ID des Benutzers wird dann der Gruppenliste hinzugefügt und der Genutzte Link gelöscht\\
     \textbf{/FS070/} Löschen der ID des Benutzers der die Gruppe verlässt aus der Gruppenliste\\
     \textbf{/FS090/} Parameter für den Zielort empfangen und in der Gruppe speichern\\
     \textbf{/FS100/} Für den Parameter 'Uhrzeit' wird genauso verfahren wie in /FS090/ \\
     \textbf{/FS110/} Position als GPS Koordinaten empfangen und zwischenspeichern\\
     \textbf{/FS120/} Bei Anfrage aktuelle GPS-Daten der Mitglieder an Mitglied senden \\
     \textbf{/FS130/} Der Datenverkehr von Server zu Client wird verschlüsselt\\
     \textbf{/FS150/} Bei Anfrage: senden der aktuellen Gruppenparameter inklusive Status der Gruppenmitglieder\\
\section{Wunschkriterien}
\subsection{Clientseitig}
     \textbf{/FC160/} Benutzung des Go-Buttons bei deaktiviertem GPS\\
     \textbf{/FC170/} Änderung des Benutzernamens nach erstmaliger Registrierung\\
     \textbf{/FC180/} Gruppenmitglieder zu Administratoren machen                \\
     \textbf{/FC190/} Änderung des Gruppennamens nach erstellen der Gruppe        \\
     \textbf{/FC200/} Benutzer nach langer Inaktivität löschen                     \\
     \textbf{/FC210/} Anzeigen der Namen von Gruppenmitgliedern, die den Go-Button schon gedrückt haben\\
     \textbf{/FC220/} Benachrichtigung über das 'Aufbrechen' anderer Gruppenmitglieder\\
     \textbf{/FC230/} Benachrichtigung bei ändern eines Treffpunkt Parameters (Ort/Uhrzeit)\\
     \textbf{/FC240/} Treffpunkte für ein spezielles Datum festlegen\\
\subsection{Serverseitig}
     \textbf{/FS170/} Einer Benutzer-ID einen neuen Namen zuordnen\\
     \textbf{/FS180/} Bei einem Benutzer in einer Gruppenliste das Admin-Flag setzen\\
     \textbf{/FS190/} Überprüfen ob Gruppenname vergeben ist und ändern\\
     \textbf{/FS200/} Regelmäßig überprüfen wann ein Nutzer zuletzt aktiv war\\
     \textbf{/FS210/} Go-Button Event von Gruppenmitglied empfangen und alle anderen Gruppenmitglieder benachrichtigen\\
     \textbf{/FS220/} Gleiches Prinzip wie beim vorherigen Punkt bei veränderung der Gruppenparameter\\
